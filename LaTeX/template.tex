%%%%%%%%%%%%%%%%%%%%%%%%%%%%%%%%%%%%%%%%%
% Medium Length Professional CV
% LaTeX Template
% Version 3.0 (December 17, 2022)
%
% This template originates from:
% https://www.LaTeXTemplates.com
%
% Author:
% Vel (vel@latextemplates.com)
%
% Original author:
% Trey Hunner (http://www.treyhunner.com/)
%
% License:
% CC BY-NC-SA 4.0 (https://creativecommons.org/licenses/by-nc-sa/4.0/)
%
%%%%%%%%%%%%%%%%%%%%%%%%%%%%%%%%%%%%%%%%%

%----------------------------------------------------------------------------------------
%	PACKAGES AND OTHER DOCUMENT CONFIGURATIONS
%----------------------------------------------------------------------------------------

\documentclass[
	a4paper, % Uncomment for A4 paper size (default is US letter)
	11pt, % Default font size, can use 10pt, 11pt or 12pt
]{resume} % Use the resume class

\usepackage{ebgaramond} % Use the EB Garamond font
\usepackage[T2A]{fontenc}

%------------------------------------------------

\name{Константин Конов - C++ developer} % Your name to appear at the top

% You can use the \address command up to 3 times for 3 different addresses or pieces of contact information
% Any new lines (\\) you use in the \address commands will be converted to symbols, so each address will appear as a single line.

\address{+7-950-948-10-91 Telegram @KonstantinKonov} % Main address

\address{kostyakonov@gmail.com} % A secondary address (optional)

\address{https://github.com/KonstantinKonov} % Contact information

%----------------------------------------------------------------------------------------

\begin{document}

%----------------------------------------------------------------------------------------
%	EDUCATION SECTION
%----------------------------------------------------------------------------------------

\begin{rSection}{Обо мне}
	\item Кандидат физ-мат наук со значительным баэкграундом в области вычислительной физики и математики
	(общий опыт работы 15 лет). В настоящее время - постдок ВШЭ.
	\item Опыт в разработке ПО 3 года. Основной язык C++, владение C и
	Python на хорошем уровне.
	\item Ищу технологическую	компанию, чей цикл разработки включает в себя матмоделирование
	и параллельное программирование.
\end{rSection}

\begin{rSection}{Образование}

	Кандидат физ.-мат. наук \hfill \textit{2016} \\
	\textbf{Казанский федеральный университет (радиофизика)} \hfill \textit{2009} \\

\end{rSection}

%----------------------------------------------------------------------------------------
%	WORK EXPERIENCE SECTION
%----------------------------------------------------------------------------------------

\begin{rSection}{Опыт работы}

	\begin{rSubsection}{МИЭМ ВШЭ (Московский институт электроники и математики им. А.Н. Тихонова Высшей школы экономики)}{2021 - н.в.}{постдок}{Москва, Россия}
		\item Моделирование физических процессов на языке CUDA C на кластере cHARISMa
		\item Применение нейросетей (Tensorflow) к задачам компьютерного зрения
		\item Обзор научной литературы и написание статей в международные рецензируемые журналы
	\end{rSubsection}

	%------------------------------------------------

	\begin{rSubsection}{КФТИ КазНЦ РAН}{2009-2021}{научный сотрудник}{Казань, Россия}
		\item Моделирование физических процессов на языке Python
		\item Построение и проведение экспериментов в области электронного парамагнитного резонанса
	\end{rSubsection}

	%------------------------------------------------

	\begin{rSubsection}{IT-лицей при Казанском федеральном университете}{2013-2014}{учитель физики}{Казань, Россия}
		\item Преподавание
	\end{rSubsection}

	%------------------------------------------------

	\begin{rSubsection}{ИХКГ СО РАН}{2011-2012}{научный сотрудник}{Новосибирск, Россия}
		\item Построение и проведение экспериментов в области молекулярной биологии
	\end{rSubsection}

	%------------------------------------------------

	\begin{rSubsection}{Казанский федеральный университет, кафедра химической физики}{2008-2009}{лаборант-исследователь}{Казань, Россия}
		\item Преподавание
	\end{rSubsection}


\end{rSection}

%----------------------------------------------------------------------------------------
%	TECHNICAL STRENGTHS SECTION
%----------------------------------------------------------------------------------------

\begin{rSection}{Технические скиллы}

	\begin{tabular}{@{} >{\bfseries}l @{\hspace{6ex}} l @{}}
		Языки программирования & C, C++, Python \\
	\end{tabular}

	\begin{tabular}{@{} >{\bfseries}l @{\hspace{6ex}} l @{}}
		Параллельное программирование и HPC & MPI, CUDA, OpenMP, Numba
	\end{tabular}

	\begin{tabular}{@{} >{\bfseries}l @{\hspace{6ex}} l @{}}
		Компиляторы, средства отладки и профилирования & g++, clang, nvcc, gprof, valgrind \\
	\end{tabular}

	\begin{tabular}{@{} >{\bfseries}l @{\hspace{6ex}} l @{}}
		Разное & \LaTeX, git, MySQL, PostgreSQL, pqxx, GSL, MatLab/GNU Octave
	\end{tabular}

\end{rSection}



%----------------------------------------------------------------------------------------
%	EXAMPLE SECTION
%----------------------------------------------------------------------------------------

%\begin{rSection}{Section Name}

%Section content\ldots

%\end{rSection}

%----------------------------------------------------------------------------------------

\end{document}
